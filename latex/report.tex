\documentclass{article}

% if you need to pass options to natbib, use, e.g.:
%     \PassOptionsToPackage{numbers, compress}{natbib}
% before loading neurips_2021

% ready for submission
\usepackage[preprint]{neurips_2021}
%\usepackage{neurips_2021}

% to compile a preprint version, e.g., for submission to arXiv, add add the
% [preprint] option:
%     \usepackage[preprint]{neurips_2021}

% to compile a camera-ready version, add the [final] option, e.g.:
%     \usepackage[final]{neurips_2021}

% to avoid loading the natbib package, add option nonatbib:
%    \usepackage[nonatbib]{neurips_2021}

\usepackage[utf8]{inputenc} % allow utf-8 input
\usepackage[T1]{fontenc}    % use 8-bit T1 fonts
\usepackage{hyperref}       % hyperlinks
\usepackage{url}            % simple URL typesetting
\usepackage{booktabs}       % professional-quality tables
\usepackage{amsfonts}       % blackboard math symbols
\usepackage{nicefrac}       % compact symbols for 1/2, etc.
\usepackage{microtype}      % microtypography
\usepackage{xcolor}         % colors

\title{Multi-label Coding Questions Classification}

% The \author macro works with any number of authors. There are two commands
% used to separate the names and addresses of multiple authors: \And and \AND.
%
% Using \And between authors leaves it to LaTeX to determine where to break the
% lines. Using \AND forces a line break at that point. So, if LaTeX puts 3 of 4
% authors names on the first line, and the last on the second line, try using
% \AND instead of \And before the third author name.

\author{%
  Tianle Wang \\
  Department of Computer Science\\
  University of Toronto\\
  \texttt{tianle.wang@mail.utoronto.ca} \\
  \And Yifan Zhao \\
  Department of Computer Science\\ 
  University of Toronto\\
  \texttt{ethany.zhao@mail.utoronto.ca} \\
  \And Yuezhexuan Zhu \\
  Department of Computer Science\\
  University of Toronto\\
  \texttt{yuezhexuan.zhu@mail.utoronto.ca} \\
}

\begin{document}

\maketitle
\begin{abstract}
Inspired by the neural language models introduced in the lecture, we have decided to design a neural network that can predict the types of algorithms and data structures required to solve a coding question. To accomplish this, we have chosen to utilize the problems from \href{https://leetcode.com/}{Leetcode} , a well-known online judge platform that provides coding questions for practicing programming skills. However, since the data size is relatively small, we plan to use techniques such as data augmentation and transfer learning to improve the accuracy of our model. Our goal is to create a model that can predict the most likely algorithm for a given coding question based on its text description. Additionally, the model will provide several top-ranking labels ordered by possibility. We will analyze multiple models, including BERT\cite{BERT} and LSTM\cite{LSTM}, to determine which model works best for our case.
\end{abstract}

\section{Introduction}
All members of our team share a common interest in solving coding questions, which we perceive as a means of enhancing our logical reasoning and programming proficiency. We have observed that the identification of an appropriate algorithm constitutes the initial and crucial step towards devising a solution. Any biased identification can lead to a series of unfortunate outcomes until the correct path is identified.

To address this issue, we have undertaken the development of a Neural Network that can effectively generate potential algorithms and data structure labels for a given coding question based on its textual description. Our proposed model is a specialized text classifier tailored to the domain of coding questions. The rationale behind this choice is twofold. Firstly, the existence of a plethora of prior research and frameworks in the field of text classification allows us to analyze various architectures and leverage pre-trained models to minimize the resources required for training while achieving superior performance. Secondly, our model seeks to enhance the accuracy and efficiency of algorithm identification, thereby facilitating the solution development process.

\section{Data Preprocessing}
Instead of fetching data from Leetcode, we decide to use the processed dataset from gzipChrist\cite{DATA} on \href{https://www.kaggle.com/}{Kaggle}. The dataset comprises 1825 problem descriptions and their corresponding required algorithms or data structures. To facilitate analysis, we developed a Python script to vectorize the labels associated with each problem. Following processing, the dataset contains 1825 rows, with the first column representing the problem description and the remaining columns indicating the presence or absence of the corresponding labels, denoted by binary values of 0 or 1.
\section{Model Architecture}

\subsection{BERT Pre-trained Model}

\subsubsection{Motivation}
\subsubsection{Design}
\subsubsection{Training}
\subsubsection{Result}

\subsection{LSTM Model}

\subsubsection{Motivation}
\subsubsection{Design}
\subsubsection{Training}
\subsubsection{Result}

\section{Comparison of Models}

\section{Limitations}
In examining our dataset, several limitations must be acknowledged to ensure a comprehensive understanding of its potential impact on the analysis. Firstly, the dataset is relatively small, as generating LeetCode problems is a complex task and the total number of problems available is less than 3,000. Secondly, our model is limited to handling text input, rendering it incapable of addressing problems that necessitate image understanding. Thirdly, the labeling of LeetCode problems is inherently biased, as the majority of coding problems are associated with major data structures, such as arrays, or widely-used algorithms like dynamic programming and greedy techniques. Lastly, the presence of mathematical formulas and special characters in problem statements may hinder the model’s tokenizer from optimally processing the input for training.


\section{Conclusion}


%\section{Citations, figures, tables, references}
%\label{others}
%
%These instructions apply to everyone.
%
%\subsection{Citations within the text}
%
%The \verb+natbib+ package will be loaded for you by default.  Citations may be
%author/year or numeric, as long as you maintain internal consistency.  As to the
%format of the references themselves, any style is acceptable as long as it is
%used consistently.
%
%The documentation for \verb+natbib+ may be found at
%\begin{center}
%  \url{http://mirrors.ctan.org/macros/latex/contrib/natbib/natnotes.pdf}
%\end{center}
%Of note is the command \verb+\citet+, which produces citations appropriate for
%use in inline text.  For example,
%\begin{verbatim}
%   \citet{hasselmo} investigated\dots
%\end{verbatim}
%produces
%\begin{quote}
%  Hasselmo, et al.\ (1995) investigated\dots
%\end{quote}
%
%If you wish to load the \verb+natbib+ package with options, you may add the
%following before loading the \verb+neurips_2021+ package:
%\begin{verbatim}
%   \PassOptionsToPackage{options}{natbib}
%\end{verbatim}
%
%If \verb+natbib+ clashes with another package you load, you can add the optional
%argument \verb+nonatbib+ when loading the style file:
%\begin{verbatim}
%   \usepackage[nonatbib]{neurips_2021}
%\end{verbatim}
%
%As submission is double blind, refer to your own published work in the third
%person. That is, use ``In the previous work of Jones et al.\ [4],'' not ``In our
%previous work [4].'' If you cite your other papers that are not widely available
%(e.g., a journal paper under review), use anonymous author names in the
%citation, e.g., an author of the form ``A.\ Anonymous.''
%
%\subsection{Footnotes}
%
%Footnotes should be used sparingly.  If you do require a footnote, indicate
%footnotes with a number\footnote{Sample of the first footnote.} in the
%text. Place the footnotes at the bottom of the page on which they appear.
%Precede the footnote with a horizontal rule of 2~inches (12~picas).
%
%Note that footnotes are properly typeset \emph{after} punctuation
%marks.\footnote{As in this example.}
%
%\subsection{Figures}
%
%\begin{figure}
%  \centering
%  \fbox{\rule[-.5cm]{0cm}{4cm} \rule[-.5cm]{4cm}{0cm}}
%  \caption{Sample figure caption.}
%\end{figure}
%
%All artwork must be neat, clean, and legible. Lines should be dark enough for
%purposes of reproduction. The figure number and caption always appear after the
%figure. Place one line space before the figure caption and one line space after
%the figure. The figure caption should be lower case (except for first word and
%proper nouns); figures are numbered consecutively.
%
%You may use color figures.  However, it is best for the figure captions and the
%paper body to be legible if the paper is printed in either black/white or in
%color.
%
%\subsection{Tables}
%
%All tables must be centered, neat, clean and legible.  The table number and
%title always appear before the table.  See Table~\ref{sample-table}.
%
%Place one line space before the table title, one line space after the
%table title, and one line space after the table. The table title must
%be lower case (except for first word and proper nouns); tables are
%numbered consecutively.
%
%Note that publication-quality tables \emph{do not contain vertical rules.} We
%strongly suggest the use of the \verb+booktabs+ package, which allows for
%typesetting high-quality, professional tables:
%\begin{center}
%  \url{https://www.ctan.org/pkg/booktabs}
%\end{center}
%This package was used to typeset Table~\ref{sample-table}.
%
%\begin{table}
%  \caption{Sample table title}
%  \label{sample-table}
%  \centering
%  \begin{tabular}{lll}
%    \toprule
%    \multicolumn{2}{c}{Part}                   \\
%    \cmidrule(r){1-2}
%    Name     & Description     & Size ($\mu$m) \\
%    \midrule
%    Dendrite & Input terminal  & $\sim$100     \\
%    Axon     & Output terminal & $\sim$10      \\
%    Soma     & Cell body       & up to $10^6$  \\
%    \bottomrule
%  \end{tabular}
%\end{table}
%
%\section{Final instructions}
%
%Do not change any aspects of the formatting parameters in the style files.  In
%particular, do not modify the width or length of the rectangle the text should
%fit into, and do not change font sizes (except perhaps in the
%\textbf{References} section; see below). Please note that pages should be
%numbered.
%
%\section{Preparing PDF files}
%
%Please prepare submission files with paper size ``US Letter,'' and not, for
%example, ``A4.''
%
%Fonts were the main cause of problems in the past years. Your PDF file must only
%contain Type 1 or Embedded TrueType fonts. Here are a few instructions to
%achieve this.
%
%\begin{itemize}
%
%\item You should directly generate PDF files using \verb+pdflatex+.
%
%\item You can check which fonts a PDF files uses.  In Acrobat Reader, select the
%  menu Files$>$Document Properties$>$Fonts and select Show All Fonts. You can
%  also use the program \verb+pdffonts+ which comes with \verb+xpdf+ and is
%  available out-of-the-box on most Linux machines.
%
%\item The IEEE has recommendations for generating PDF files whose fonts are also
%  acceptable for NeurIPS. Please see
%  \url{http://www.emfield.org/icuwb2010/downloads/IEEE-PDF-SpecV32.pdf}
%
%\item \verb+xfig+ "patterned" shapes are implemented with bitmap fonts.  Use
%  "solid" shapes instead.
%
%\item The \verb+\bbold+ package almost always uses bitmap fonts.  You should use
%  the equivalent AMS Fonts:
%\begin{verbatim}
%   \usepackage{amsfonts}
%\end{verbatim}
%followed by, e.g., \verb+\mathbb{R}+, \verb+\mathbb{N}+, or \verb+\mathbb{C}+
%for $\mathbb{R}$, $\mathbb{N}$ or $\mathbb{C}$.  You can also use the following
%workaround for reals, natural and complex:
%\begin{verbatim}
%   \newcommand{\RR}{I\!\!R} %real numbers
%   \newcommand{\Nat}{I\!\!N} %natural numbers
%   \newcommand{\CC}{I\!\!\!\!C} %complex numbers
%\end{verbatim}
%Note that \verb+amsfonts+ is automatically loaded by the \verb+amssymb+ package.
%
%\end{itemize}
%
%If your file contains type 3 fonts or non embedded TrueType fonts, we will ask
%you to fix it.
%
%\subsection{Margins in \LaTeX{}}
%
%Most of the margin problems come from figures positioned by hand using
%\verb+\special+ or other commands. We suggest using the command
%\verb+\includegraphics+ from the \verb+graphicx+ package. Always specify the
%figure width as a multiple of the line width as in the example below:
%\begin{verbatim}
%   \usepackage[pdftex]{graphicx} ...
%   \includegraphics[width=0.8\linewidth]{myfile.pdf}
%\end{verbatim}
%See Section 4.4 in the graphics bundle documentation
%(\url{http://mirrors.ctan.org/macros/latex/required/graphics/grfguide.pdf})
%
%A number of width problems arise when \LaTeX{} cannot properly hyphenate a
%line. Please give LaTeX hyphenation hints using the \verb+\-+ command when
%necessary.
%
%\begin{ack}
%Use unnumbered first level headings for the acknowledgments. All acknowledgments
%go at the end of the paper before the list of references. Moreover, you are required to declare
%funding (financial activities supporting the submitted work) and competing interests (related financial activities outside the submitted work).
%More information about this disclosure can be found at: \url{https://neurips.cc/Conferences/2021/PaperInformation/FundingDisclosure}.
%
%Do {\bf not} include this section in the anonymized submission, only in the final paper. You can use the \texttt{ack} environment provided in the style file to autmoatically hide this section in the anonymized submission.
%\end{ack}

%\section*{References}
{
\small
\bibliographystyle{plain}
\begin{thebibliography}{9}
\bibitem{BERT}
Devlin, J., Chang, M., Lee, K., \& Toutanova, K. (2018). BERT: Pre-training of Deep Bidirectional Transformers for Language Understanding. ArXiv. /abs/1810.04805.
\bibitem{LSTM}
S. Hochreiter and J. Schmidhuber, "Long Short-Term Memory," in Neural Computation, vol. 9, no. 8, pp. 1735-1780, 15 Nov. 1997, doi: 10.1162/neco.1997.9.8.1735.
\bibitem{DATA}
gzipChrist. (2023, February). Leetcode Problem Dataset, Version 8.82. Retrieved April 10, 2023 from https://www.kaggle.com/datasets/gzipchrist/leetcode-problem-dataset.
\end{thebibliography}
}
%%%%%%%%%%%%%%%%%%%%%%%%%%%%%%%%%%%%%%%%%%%%%%%%%%%%%%%%%%%%
%\section*{Checklist}
%
%%%% BEGIN INSTRUCTIONS %%%
%The checklist follows the references.  Please
%read the checklist guidelines carefully for information on how to answer these
%questions.  For each question, change the default \answerTODO{} to \answerYes{},
%\answerNo{}, or \answerNA{}.  You are strongly encouraged to include a {\bf
%justification to your answer}, either by referencing the appropriate section of
%your paper or providing a brief inline description.  For example:
%\begin{itemize}
%  \item Did you include the license to the code and datasets? \answerYes{See Section~\ref{gen_inst}.}
%  \item Did you include the license to the code and datasets? \answerNo{The code and the data are proprietary.}
%  \item Did you include the license to the code and datasets? \answerNA{}
%\end{itemize}
%Please do not modify the questions and only use the provided macros for your
%answers.  Note that the Checklist section does not count towards the page
%limit.  In your paper, please delete this instructions block and only keep the
%Checklist section heading above along with the questions/answers below.
%%%% END INSTRUCTIONS %%%
%
%\begin{enumerate}
%
%\item For all authors...
%\begin{enumerate}
%  \item Do the main claims made in the abstract and introduction accurately reflect the paper's contributions and scope?
%    \answerTODO{}
%  \item Did you describe the limitations of your work?
%    \answerTODO{}
%  \item Did you discuss any potential negative societal impacts of your work?
%    \answerTODO{}
%  \item Have you read the ethics review guidelines and ensured that your paper conforms to them?
%    \answerTODO{}
%\end{enumerate}
%
%\item If you are including theoretical results...
%\begin{enumerate}
%  \item Did you state the full set of assumptions of all theoretical results?
%    \answerTODO{}
%	\item Did you include complete proofs of all theoretical results?
%    \answerTODO{}
%\end{enumerate}
%
%\item If you ran experiments...
%\begin{enumerate}
%  \item Did you include the code, data, and instructions needed to reproduce the main experimental results (either in the supplemental material or as a URL)?
%    \answerTODO{}
%  \item Did you specify all the training details (e.g., data splits, hyperparameters, how they were chosen)?
%    \answerTODO{}
%	\item Did you report error bars (e.g., with respect to the random seed after running experiments multiple times)?
%    \answerTODO{}
%	\item Did you include the total amount of compute and the type of resources used (e.g., type of GPUs, internal cluster, or cloud provider)?
%    \answerTODO{}
%\end{enumerate}
%
%\item If you are using existing assets (e.g., code, data, models) or curating/releasing new assets...
%\begin{enumerate}
%  \item If your work uses existing assets, did you cite the creators?
%    \answerTODO{}
%  \item Did you mention the license of the assets?
%    \answerTODO{}
%  \item Did you include any new assets either in the supplemental material or as a URL?
%    \answerTODO{}
%  \item Did you discuss whether and how consent was obtained from people whose data you're using/curating?
%    \answerTODO{}
%  \item Did you discuss whether the data you are using/curating contains personally identifiable information or offensive content?
%    \answerTODO{}
%\end{enumerate}
%
%\item If you used crowdsourcing or conducted research with human subjects...
%\begin{enumerate}
%  \item Did you include the full text of instructions given to participants and screenshots, if applicable?
%    \answerTODO{}
%  \item Did you describe any potential participant risks, with links to Institutional Review Board (IRB) approvals, if applicable?
%    \answerTODO{}
%  \item Did you include the estimated hourly wage paid to participants and the total amount spent on participant compensation?
%    \answerTODO{}
%\end{enumerate}
%
%\end{enumerate}

%%%%%%%%%%%%%%%%%%%%%%%%%%%%%%%%%%%%%%%%%%%%%%%%%%%%%%%%%%%%

\appendix

\section{Appendix}

Optionally include extra information (complete proofs, additional experiments and plots) in the appendix.
This section will often be part of the supplemental material.

\end{document}
